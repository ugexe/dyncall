%//////////////////////////////////////////////////////////////////////////////
%
% Copyright (c) 2012-2022 Daniel Adler <dadler@uni-goettingen.de>,
%                         Tassilo Philipp <tphilipp@potion-studios.com>
%
% Permission to use, copy, modify, and distribute this software for any
% purpose with or without fee is hereby granted, provided that the above
% copyright notice and this permission notice appear in all copies.
%
% THE SOFTWARE IS PROVIDED "AS IS" AND THE AUTHOR DISCLAIMS ALL WARRANTIES
% WITH REGARD TO THIS SOFTWARE INCLUDING ALL IMPLIED WARRANTIES OF
% MERCHANTABILITY AND FITNESS. IN NO EVENT SHALL THE AUTHOR BE LIABLE FOR
% ANY SPECIAL, DIRECT, INDIRECT, OR CONSEQUENTIAL DAMAGES OR ANY DAMAGES
% WHATSOEVER RESULTING FROM LOSS OF USE, DATA OR PROFITS, WHETHER IN AN
% ACTION OF CONTRACT, NEGLIGENCE OR OTHER TORTIOUS ACTION, ARISING OUT OF
% OR IN CONNECTION WITH THE USE OR PERFORMANCE OF THIS SOFTWARE.
%
%//////////////////////////////////////////////////////////////////////////////

\subsection{SPARC64 Calling Conventions}

\paragraph{Overview}

The SPARC family of processors is based on the SPARC instruction set architecture, which comes in basically three revisions,
V7, V8\cite{SPARCV8}\cite{SPARCSysV}\cite{SPARCCD} and V9\cite{SPARCV9}\cite{SPARCV9SysV}\cite{SPARCCD}. The former two are 32-bit (see previous chapter) whereas the latter refers to the 64-bit SPARC architecture.
SPARC uses big endian byte order, however, V9 supports also little endian byte order, but for data access only, not instruction access.\\
\\
There are two proposals, one from Sun and one from Hal, which disagree on how to handle some aspects of this calling convention.\\

\paragraph{\product{dyncall} support}

\product{dyncall} fully supports the SPARC 64-bit instruction set (V9), for calls and callbacks.

\subsubsection{SPARC (64-bit) Calling Convention}

\begin{itemize}
\item 32 double precision floating point registers (d0,d2,d4,...,d62, usable as 16 quad precision ones q0,q4,q8,...g60, and also first half of them are usable as 32 single precision registers f0-f31)
\item 32 64-bit integer/pointer registers out of a bigger (vendor/model dependent) number that are accessible at a time (8 are global ones (g*), whereas the remaining 24 form a register window with 8 input (i*), 8 output (o*) and 8 local (l*) ones)
\item calling a function shifts the register window, the old output registers become the new input registers (old local and input ones are not accessible anymore)
\item stack and frame pointer are offset by a BIAS of 2047 (see official doc for reasons)
\end{itemize}

\begin{table}[h]
\begin{tabular*}{0.95\textwidth}{lll}
Name                          & Alias                          & Brief description\\
\hline
{\bf \%g0}                    & \%r0                           & Read-only, hardwired to 0 \\
{\bf \%g1-\%g7}               & \%r1-\%r7                      & Global \\
{\bf \%o0-\%o3 and \%i0-\%i3} & \%r8-\%r11 and \%r24-\%r27     & Output and input argument registers, return value \\
{\bf \%o4,\%o5 and \%i4,\%i5} & \%r12,\%r13 and \%r28,\%r29    & Output and input argument registers \\
{\bf \%o6 and \%i6}           & \%r14 and \%r30, \%sp and \%fp & Stack and frame pointer (NOTE, offset with a BIAS of 2047) \\
{\bf \%o7 and \%i7}           & \%r15 and \%r31                & Return address (caller writes to o7, callee uses i7) \\
{\bf \%l0-\%l7}               & \%r16-\%r23                    & preserve \\
{\bf \%d0,\%d2,\%d4,\%d6}     &                                & scratch, Floating point arguments, return value \\
{\bf \%d8,\%d10,...,\%d14}    &                                & scratch, Floating point arguments \\
{\bf \%d16,\%d18,...,\%d30}   &                                & scratch (preserve for Hal), Floating point arguments \\
{\bf \%d32,\%d34,...,\%d62}   &                                & scratch (preserve for Hal) \\
\end{tabular*}
\caption{Register usage on sparc64 calling convention}
\end{table}

\paragraph{Parameter passing}
\begin{itemize}
\item stack grows down
\item stack parameter order: right-to-left
\item caller cleans up the stack
\item stack frame is always aligned to 16 bytes
\item first 6 integers are passed in registers using \%o0-\%o5
\item first 8 quad precision floating point args (or 16 double precision, or 32 single precision) are passed in floating point registers (\%q0,\%q4,...,\%q28 or \%d0,\%d2,...,\%d30 or \%f0-\%f31, respectively)
\item for every other argument the stack is used
\item single precision floating point args are passed in odd \%f* registers, and are "right aligned" in their 8-byte space on the stack
\item for every argument passed, corresponding \%o*, \%f* register or stack space is skipped (e.g. passing a double as 3rd call argument, \%d4 is used and \%o2 is skipped)
\item all arguments \textless=\ 64 bit are passed as 64 bit values
\item minimum stack size is 128 bytes, b/c stack pointer must always point at enough space to store all \%i* and \%l* registers, used when running out of register windows
\item if needed, register spill area (both, integer and float arguments are spilled in order) is adjacent to parameters
\item aggregates (struct, union) \textless=\ 16 bytes are passed field-by-field, {\bf however} evaluated as a sequence of 8-byte parameter slots
\begin{itemize}
\item fields are left justified in register or stack slots
\item integers in a slot are passed as such (either via \%o* registers or the stack)
\item single precision floats (using half of the slot) use even numbered \%f* registers when they occupy the left half, odd numbered ones otherwise (no register skipping logic applied within a slot)
\item splitting aggregates between registers and stack is allowed
\end{itemize}
\item aggregates (struct, union) and types \textgreater\ 16 bytes are passed indirectly, as a pointer to a correctly aligned copy of the data (that copy can be avoided under certain conditions)
% from spec:
%Structure or union types up to eight bytes in size are assigned to one parameter array word, and align to eight-byte
%boundaries.
%Structure or union types larger than eight bytes, and up to sixteen bytes in size are assigned to two consecutive
%parameter array words, and align according to the alignment requirements of the structure or at least to an eight-byte
%boundary.
%Structure or union types are always left-justified, whether stored in registers or memory. The individual fields of a
%structure (or containing storage unit in the case of bit fields) are subject to promotion into registers based on their type
%using the same rules as apply to scalar values (with the addition that a single-precision floating-point number assigned
%to the left half of an argument slot will be promoted into the corresponding even-numbered float register.). Any union
%type being passed directly is subject to promotion into the appropriate integer register(s).
%Note that a sixteen-byte structure with all integral fields assigned to locations %sp+BIAS+168 and %sp+BIAS+176 will
%be “split,” as the contents of location %sp+BIAS+168 will be promoted to %o5.
%Structures or unions larger than sixteen bytes are copied by the caller and passed indirectly; the caller will pass the
%address of a correctly aligned structure value. This sixty-four bit address will occupy one word in the parameter array,
%and may be promoted to an %o register like any other pointer value. The callee may modify the addressed structure.
%The caller can omit the copy if such omission cannot be detected. That requires (at least) that:
%* the original aggregate is already properly aligned,
%* the original aggregate is not aliased,
%* the original aggregate is not used after the call, and
%* no language-specific semantics require the copy.
\end{itemize}

\paragraph{Return values}

\begin{itemize}
\item results are expected by caller to be returned in \%o0-\%o3 (after reg window restore, meaning callee writes to \%i0-\%i3) for integers
\item \%d0,\%d2,\%d4,\%d6 are used for floating point values
\item the fields of aggregates (struct, union) \textless 32 bytes are returned via registers registers mentioned above (which are
assigned following the same logic as when passing the aggregate as a first argument to a function)
\item aggregates (struct, union) \textgreater= 32 bytes are returned in a space allocated by the caller, with a pointer to it
passed as first parameter to the function called (meaning in \%o0)
% from spec:
%Structure and union return types up to thirty-two bytes in size are returned in registers. The registers are assigned as if
%the value was being passed as the first argument to a function with a known prototype.
%For types with a larger size the caller allocates an area large enough and aligned properly to hold the return value, and
%passes a pointer to that area as an implicit first argument (of type pointer-to-data) to the callee. This implicit argument
%logically precedes the first actual argument, and is allocated according to normal argument passing rules (i.e. into %o0).
%The callee must store the function return value in the result area before control is returned to the caller and after the last
%use or definition of any variable that might overlap with the result area. If the callee is terminated through any means
%other than a normal function return (e.g., through a call to the longjmp function), the contents of the result area are
%undefined.
%In the common case that the caller immediately assigns the returned value to a program variable, the caller may
%substitute the address of the assigned program variable in place of the allocated result area and omit the code to do the
%assignment, as long as this substitution does not change the program’s externally visible behavior.
%Note also that the caller is required to provide the implicit argument and a properly sized and aligned receiving area
%even if it does not wish to use the callee’s function result. In that case, the caller may simply pass a pointer to a scratch
%area.
%So that compilers are not forced to emit in-line code for structure copy, Section 6.2 defines a set of routines optimized
%for this purpose. In the case of a routine which had kept its first argument in %i0 and was returning a value pointed to
%by %i1, epilogue code would take the form:
%
%mov %i0, %o0
%mov %i1, %o1
%call __align_cpy_n
%mov size, %o2
%ret
%restore %o0, %g0, %o0
%
\end{itemize}

\paragraph{Stack layout}

% verified/amended: TP nov 2019 (see also doc/disas_examples/sparc64.sparc64.disas)
Stack directly after function prolog:\\

\begin{figure}[h]
\begin{tabular}{5|3|1 1}
                                   & \vdots                      &                                &                               \\
\hhline{~=~~}
local data (and padding)           & \hspace{4cm}                &                                & \mrrbrace{8}{caller's frame}  \\
\hhline{~-~~}
\mrlbrace{6}{parameter area}       & arg n-1                     & \mrrbrace{3}{stack parameters} &                               \\
                                   & \ldots                      &                                &                               \\
                                   & arg 6                       &                                &                               \\
                                   & \%o5                        & \mrrbrace{3}{spill area}       &                               \\
                                   & \ldots                      &                                &                               \\
                                   & \%o0                        &                                &                               \\
\hhline{~-~~}
register save area (\%i* and \%l*) &                             &                                &                               \\
\hhline{~=~~}
local data (and padding)           &                             &                                & \mrrbrace{3}{current frame}   \\
\hhline{~-~~}
parameter area                     &                             &                                &                               \\
\hhline{~-~~}
                                   & \vdots                      &                                &                               \\
\end{tabular}
\caption{Stack layout on sparc64 calling convention}
\end{figure}

