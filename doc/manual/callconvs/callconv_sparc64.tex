%//////////////////////////////////////////////////////////////////////////////
%
% Copyright (c) 2012-2017 Daniel Adler <dadler@uni-goettingen.de>,
%                         Tassilo Philipp <tphilipp@potion-studios.com>
%
% Permission to use, copy, modify, and distribute this software for any
% purpose with or without fee is hereby granted, provided that the above
% copyright notice and this permission notice appear in all copies.
%
% THE SOFTWARE IS PROVIDED "AS IS" AND THE AUTHOR DISCLAIMS ALL WARRANTIES
% WITH REGARD TO THIS SOFTWARE INCLUDING ALL IMPLIED WARRANTIES OF
% MERCHANTABILITY AND FITNESS. IN NO EVENT SHALL THE AUTHOR BE LIABLE FOR
% ANY SPECIAL, DIRECT, INDIRECT, OR CONSEQUENTIAL DAMAGES OR ANY DAMAGES
% WHATSOEVER RESULTING FROM LOSS OF USE, DATA OR PROFITS, WHETHER IN AN
% ACTION OF CONTRACT, NEGLIGENCE OR OTHER TORTIOUS ACTION, ARISING OUT OF
% OR IN CONNECTION WITH THE USE OR PERFORMANCE OF THIS SOFTWARE.
%
%//////////////////////////////////////////////////////////////////////////////

\subsection{SPARC64 Calling Convention}

\paragraph{Overview}

The SPARC family of processors is based on the SPARC instruction set architecture, which comes in basically tree revisions,
V7, V8\cite{SPARCV8}\cite{SPARCSysV} and V9\cite{SPARCV9}\cite{SPARCV9SysV}. The former two are 32-bit (see previous chapter) whereas the latter refers to the 64-bit SPARC architecture.
SPARC uses big endian byte order, however, V9 supports also little endian byte order, but for data access only, not instruction access.\\
\\
There are two probosals, one from Sun and one from Hal, which disagree on how to handle some aspects of this calling convention.\\

\paragraph{\product{dyncall} support}

\product{dyncall} fully supports the SPARC 64-bit instruction set (V9), for calls and callbacks.

\subsubsection{SPARC (64-bit) Calling Convention}

\begin{itemize}
\item 32 double precision floating point registers (d0,d2,d4,...,d62, usable as 16 quad precision ones q0,q4,q8,...g60, and also first half of them are usable as 32 single precision registers f0-f31)
\item 32 64-bit integer/pointer registers out of a bigger (vendor/model dependent) number that are accessible at a time (8 are global ones (g*), whereas the remaining 24 form a register window with 8 input (i*), 8 output (o*) and 8 local (l*) ones)
\item calling a function shifts the register window, the old output registers become the new input registers (old local and input ones are not accessible anymore)
\item stack and frame pointer are offset by a BIAS of 2047 (see official doc for reasons)
\end{itemize}

\begin{table}[h]
\begin{tabular*}{0.95\textwidth}{lll}
Name                          & Alias                          & Brief description\\
\hline
{\bf \%g0}                    & \%r0                           & Read-only, hardwired to 0 \\
{\bf \%g1-\%g7}               & \%r1-\%r7                      & Global \\
{\bf \%o0-\%o3 and \%i0-\%i3} & \%r8-\%r11 and \%r24-\%r27     & Output and input argument registers, return value \\
{\bf \%o4,\%o5 and \%i4,\%i5} & \%r12,\%r13 and \%r28,\%r29    & Output and input argument registers \\
{\bf \%o6 and \%i6}           & \%r14 and \%r30, \%sp and \%fp & Stack and frame pointer (NOTE, value is pointing to stack/frame minus a BIAS of 2047) \\
{\bf \%o7 and \%i7}           & \%r15 and \%r31                & Return address (caller writes to o7, callee uses i7) \\
{\bf \%l0-\%l7}               & \%r16-\%r23                    & preserve \\
{\bf \%d0,\%d2,\%d4,\%d6}     &                                & Floating point arguments, return value \\
{\bf \%d8,\%d10,...,\%d30}    &                                & Floating point arguments \\
{\bf \%d32,\%d36,...,\%d62}   &                                & scratch (but, according do Hal, \%d16,...,\%d46 are preserved) \\
\end{tabular*}
\caption{Register usage on sparc64 calling convention}
\end{table}

\paragraph{Parameter passing}
\begin{itemize}
\item stack grows down
\item stack parameter order: right-to-left
\item caller cleans up the stack
\item stack frame is always aligned to 16 bytes
\item first 6 integers are passed in registers using \%o0-\%o5
\item first 8 quad precision floating point args (or 16 double precision, or 32 single precision) are passed in floating point registers (\%q0,\%q4,...,\%q28 or \%d0,\%d2,...,\%d30 or \%f0-\%f32, respectively)
\item for every other argument the stack is used
\item single precision floating point args are passed in odd \%f* registers, and are "right aligned" in their 8-byte space on the stack
\item for every argument passed, corresponding \%o*, \%f* register or stack space is skipped (e.g. passing a doube as 3rd call argument, \%d4 is used and \%o2 is skipped)
\item all arguments \textless=\ 64 bit are passed as 64 bit values
\item minimum stack size is 128 bytes, b/c stack pointer must always point at enough space to store all \%i* and \%l* registers, used when running out of register windows
\item if needed, register spill area (for integer arguments passed via \%o0-\%o5) is adjacent to parameters
\item results are expected by caller to be returned in \%o0-\%o3 (after reg window restore, meaning callee writes to \%i0-\%i3) for integers, \%d0,\%d2,\%d4,\%d6 for floats
\item structs/unions up to 32b, the fields are returned via the respective registers mentioned in the previous bullet point
\item for structs/unions \textgreater= 32b, the caller allocates the space and a pointer to it is passed as hidden first parameter to the function called (meaning in \%o0)
\end{itemize}

\paragraph{Stack layout}

Stack directly after function prolog:\\

\begin{figure}[h]
\begin{tabular}{5|3|1 1}
\hhline{~-~~}
                                   & \vdots                      &                                &                               \\
\hhline{~=~~}
local data (and padding)           & \hspace{4cm}                &                                & \mrrbrace{8}{caller's frame}  \\
\hhline{~-~~}
\mrlbrace{7}{parameter area}       & argument x                  & \mrrbrace{3}{stack parameters} &                               \\
                                   & \ldots                      &                                &                               \\
                                   & argument 6                  &                                &                               \\
                                   & input argument 5 spill      & \mrrbrace{3}{spill area}       &                               \\
                                   & \ldots                      &                                &                               \\
                                   & input argument 0 spill      &                                &                               \\
\hhline{~-~~}
register save area (\%i* and \%l*) &                             &                                &                               \\
\hhline{~=~~}
local data (and padding)           &                             &                                & \mrrbrace{3}{current frame}   \\
\hhline{~-~~}
parameter area                     &                             &                                &                               \\
\hhline{~-~~}
                                   & \vdots                      &                                &                               \\
\hhline{~-~~}
\end{tabular}
\\
\\
\\
\caption{Stack layout on sparc64 calling convention}
\end{figure}

