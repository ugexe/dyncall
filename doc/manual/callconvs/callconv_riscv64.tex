%//////////////////////////////////////////////////////////////////////////////
%
% Copyright (c) 2024 Tassilo Philipp <tphilipp@potion-studios.com>
%
% Permission to use, copy, modify, and distribute this software for any
% purpose with or without fee is hereby granted, provided that the above
% copyright notice and this permission notice appear in all copies.
%
% THE SOFTWARE IS PROVIDED "AS IS" AND THE AUTHOR DISCLAIMS ALL WARRANTIES
% WITH REGARD TO THIS SOFTWARE INCLUDING ALL IMPLIED WARRANTIES OF
% MERCHANTABILITY AND FITNESS. IN NO EVENT SHALL THE AUTHOR BE LIABLE FOR
% ANY SPECIAL, DIRECT, INDIRECT, OR CONSEQUENTIAL DAMAGES OR ANY DAMAGES
% WHATSOEVER RESULTING FROM LOSS OF USE, DATA OR PROFITS, WHETHER IN AN
% ACTION OF CONTRACT, NEGLIGENCE OR OTHER TORTIOUS ACTION, ARISING OUT OF
% OR IN CONNECTION WITH THE USE OR PERFORMANCE OF THIS SOFTWARE.
%
%//////////////////////////////////////////////////////////////////////////////

\subsection{RISC-V 64 Calling Conventions}

\paragraph{Overview}

RISC-V 64 uses LP64 as programming model (64-bit pointers and long integers),
the word size is 32 bits (instructions with 'd' mnemonic indicate 64bit
operations, 'w' 32bit, 'h' 16bit and 'b' 8bit). It has a modular design, with a
base ISA and optional extensions. Two default calling conventions are specified
in \cite{riscvcc}, a standard one (called \emph{RVG calling convention}) and a
soft-float convention for implementations lacking the floating point extension.

RISC-V 64 features thirty-two 64 bit registers (16 in the embedded RISC-V
variant), namely {\bf x0-x31}, where x0 is a zero register and the others are
general purpose registers.
With the floating point extension, there is an additional set of thirty-two 64
bit floating point registers, {\bf f0-f31}.

Each register also has a symbolic name which illustrates the calling
conventions' intended purpose.

\paragraph{\product{dyncall} support}

The \product{dyncall} library supports the RISC-V 64-bit hardware- and software float ABI, for both, calls and callbacks.

\subsubsection{RVG calling convention (hard float)}

\paragraph{Registers and register usage}

Up to eight integer and floating point registers are used to pass parameters.

\begin{table}[h]
\begin{tabular*}{0.95\textwidth}{lll}
Name          & Symbolic Name & Brief description\\
\hline
{\bf x0}      & {\bf zero}     & hard-wired, always zero\\
{\bf x1}      & {\bf ra}       & permanent, return address / link register\\
{\bf x2}      & {\bf sp}       & permanent, stack pointer\\
{\bf x3}      & {\bf gp}       & scratch, global pointer\\
{\bf x4}      & {\bf tp}       & scratch, thread pointer\\
{\bf x5-x7}   & {\bf t0-t2}    & scratch, temporaries\\
{\bf x8}      & {\bf s0/fp}    & permanent, saved register 0 / frame pointer\\
{\bf x9}      & {\bf s1}       & permanent, saved register 1\\
{\bf x10-x11} & {\bf a0-a1}    & scratch, int parameters, return value\\
{\bf x12-x17} & {\bf a2-a7}    & scratch, int parameters\\
{\bf x18-x27} & {\bf s2-s11}   & permanent, saved registers\\
{\bf x28-x31} & {\bf t3-t6}    & scratch, temporaries\\
{\bf f0-f7}   & {\bf ft0-ft7}  & scratch, fp temporaries\\
{\bf f8-f9}   & {\bf fs0-fs1}  & permanent, saved fp registers\\
{\bf f10-f11} & {\bf fa0-fa1}  & scratch, fp parameters, fp return value\\
{\bf f12-f17} & {\bf fa2-fa7}  & scratch, fp parameters\\
{\bf f18-f27} & {\bf fs2-fs11} & permanent, saved fp registers\\
{\bf f28-f31} & {\bf ft8-ft11} & scratch, fp temporaries\\

\end{tabular*}
\caption{Register usage on riscv64}
\end{table}

\paragraph{Parameter passing}

\begin{itemize}
\item stack parameter order: right-to-left
\item caller cleans up the stack
\item stack grows down
\item first 8 integer arguments are passed using x10-x17/a0-a7
\item first 8 floating point arguments are passed using f10-f17/fa0-fa7
\item 128 bit arguments passed via registers are passed in even-odd register pairs (skipping a register, if necessary, which remains unused)
\item aggregates (struct, union) \textless=\ 16 bytes in size are passed via registers (but only in float registers if members are either one or two floating point non-union and non-array members)
\item arguments \textgreater 128 bits are passed as a pointer pointing to the argument (pointer to a copy, if needed)
\item subsequent parameters are pushed onto the stack
\item if the callee takes the address of one of the parameters and uses it to address other parameters (e.g. varargs) it has to copy - in its prolog - the first 8 integer
and 8 floating-point registers to a reserved stack area adjacent to the other parameters on the stack (only the unnamed integer parameters require saving, though)
\item unnamed floating-point arguments to variadic functions are passed in integer registers
\item {\it non-trivial} C++ aggregates (as defined by the language) of any size, are passed indirectly via a pointer to a copy of the aggregate
\item stack is required throughout to be 16-byte aligned
\end{itemize}

\paragraph{Return values}

\begin{itemize}
\item integer return values use x10-x11/a0-a1
\item floating-point return values use f10-f11/fa0-fa1
\item aggregates (struct, union) that would be passed via registers if passed as a first param, are returned via those registers
\item for {\it non-trivial} C++ aggregates, the caller allocates space, passes pointer to it to the callee via a0 as hidden param, and callee writes return value to this space
\item for aggregates not returnable via registers (e.g. \textgreater\ 16b, ...), the caller allocates space, passes pointer to it to the callee through
a0 as hidden param, and callee writes return value to this space
\end{itemize}

\paragraph{Stack layout}

% verified: TP feb 2024 (see also doc/disas_examples/riscv64.rvg.disas)
Stack directly after function prolog:\\

\begin{figure}[h]
\begin{tabular}{5|3|1 1}
                                         & \vdots                         &                                      &                              \\
\hhline{~=~~}
register save area                       & \hspace{4cm}                   &                                      & \mrrbrace{5}{caller's frame} \\
\hhline{~-~~}
local data                               &                                &                                      &                              \\
\hhline{~-~~}
\mrlbrace{7}{parameter area}             & arg n-1                        & \mrrbrace{3}{stack parameters}       &                              \\
                                         & \ldots                         &                                      &                              \\
                                         & first arg passed on stack      &                                      &                              \\
\hhline{~=~~}
                                         & a? or fa?                      & \mrrbrace{4}{spill area (if needed)} & \mrrbrace{7}{current frame}  \\
                                         & \ldots                         &                                      &                              \\
                                         & first unnamed reg args         &                                      &                              \\
                                         & named reg args (optional)      &                                      &                              \\
\hhline{~-~~}
register save area (with return address) &                                &                                      &                              \\ % fp will point here (to 1st arg)
\hhline{~-~~}
local data                               &                                &                                      &                              \\
\hhline{~-~~}
parameter area                           & \vdots                         &                                      &                              \\
\end{tabular}
\caption{Stack layout on riscv64}
\end{figure}

\clearpage


\subsubsection{Soft-Float calling convention}

The software floating point calling convention is the same as the RVG
convention, except that no floating point registers are available, and
floating-point parameters are passed and returned in integer registers,
following the rules for integer parameters of the same size.

